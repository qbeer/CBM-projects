\documentclass[a4paper,12pt]{article}
\usepackage[utf8]{inputenc}
\usepackage[T1]{fontenc}
\usepackage[hungarian]{babel}
\usepackage{graphicx}
\usepackage{geometry}
\geometry{a4paper,
		     tmargin = 2mm, 
		     lmargin = 2mm,
		     rmargin = 2mm,
		     bmargin = 2mm}
\usepackage{mathtools}
\usepackage{amsmath}
\usepackage{color}
\usepackage{setspace}
\usepackage{amsmath,amssymb}
\usepackage{float}
\usepackage{indentfirst}

\linespread{1.2}

\begin{document}
\section{QCD a CBM-ben}
\par A kvantumszíndinamika elemi részecskéi a kvarkok és antikvarkok, amelyek a gluonokon keresztül hatnak kölcsön. Ezek mindegyike színtöltést hordoz, ahol a szín is kvantumszám. A gluonoknak 8 fajtája van, hogy az összes színátmenetet le lehessen velük írni.  A gluonok önmagukkal is kölcsön tudnak hatni.
\par A CBM detektor elsődleges célja, hogy a barionikus anyag fázisátalakulásait vizsgálja. A fázisdiagram pontos feltérképezése elengedhetetlen ahhoz, hogy az arról alkotott elméleteket alátámasszuk.
\par A kvark-gluon plazma a barionikus anyag egyik fázisa. Ez az állapot jelen volt a Nagy Bummnál (kis sűrűség, extrém hőmérséklet) valamint minden valószínűség szerint a neutron csillagok magjában is megtalálható (kis hőmérséklet, extrém sűrűség).  A kvark-gluon plazma földi megfigyelésének egyetlen lehetőségét a nagy energiás részecskegyorsítók biztosítják. 
\par Ehhez szükséges gyorsítók és detektorok több helyen is léteznek a világon, RHIC, LHC, de a FAIR projekt sajátossága, hogy ezeknél magasabb barionsűrűséget akar elérni, valamint pont olyan enegián dolgozik majd, hogy képes legyen az elsőrendű fázisátlakulás vizsgálatára, valamint a kritikus pont környékének feltérképezésére.
\par Az $up$ és $down$ kvarkok tömege nagy energiás közelítésben nagyjából nulla, ezért szokták azt mondani, hogy a QCD-nek viszonylagos királis szimmetriája van. Ez alacsony hőmérsékleteken (és sűrűségeken sérülhet), ezt nevezzük királis szimmetriasértésnek. Egy-egy ütközés során nagyjából $10^{-22}$s-ig tartó állapotra koncentrálunk, ekkor van jelen ugyanis kvark-gluon plazma. A jelenség tranzienssége miatt természetesen csak a melléktermékek, a részecskezápor vizsgálható. [ fázisdiagramok: 2 ]
\section{A CBM detektor}
\par A detektor elrendezése balról jobbra haladva a következő (ábra):
\begin{itemize}
	\item CBM szupravezető mágnes szilícium spektrométerrel
	\item a micro vertex detektor ( MVD ) az előbbi belsejében
	\item a szilícium követő rendszer ( STS ) is
	\item Cserenkov-detektor ( RICH - ring imaging Cherenkov detector - világos kék )
	\item müon sprektrométer ( fekete )
	\item ezt követi 4 réteg átmeneti sugárzás ( TRD - transition radiation detector ) detektor
	\item és egy time-of-flight ( TOF ) fal
	\item a fő detektorok után található még  egy célfigyelő detektor (PSD)  
\end{itemize}
\par A szilícium követő rendszer feladata az, hogy rekonstruálja majd a részecskék trajektóriáit. Csak töltött részecskék észlelésére képes, 
de képes mérni a töltés nagyságát és az impulzust is tud mérni. A TOF fal igen nagy felbontást tud elérni, nagyjából 60 ps-os felbontásra is
lehetőség van.
\par A CBM projekt még egyenlőre csak terv szintjén létezik, a szimulációt folyamatosan fejlesztik. Jövőre, vagy legkésőbb 2019-re már tervben 
van egy miniCBM detektor építése az esetleg később felmerülő tervezési, kivitelezési problémák elkerülésére. A FAIR létesítmény építése idén 
nyáron kezdődött és az első részecskenyaláb 2022-ben várható. A miniCBM projekt a meglévő GSI gyorsítónál fog tevékenykedni az addig fennmaradó
időben, ahol megpróbálják a számítógépfarmot tökéletesíteni, hogy az adatokat minél gyorsabban feldolgozhassák.
\par A FAIR tudósai kifejlesztettek egy több tízezer soros szimulációt, ami a ROOT-on alapszik. Ezt ők cbmROOT-nak hívják, mivel teljes
egészében a CBM-hez igazodik és ingyenesen elérhető bárki számára. Sok jól ismert nehézion szimulációs eljárást használnak, amik a CBM
környezetre vannak szabva, úgy mint: UrQMD ( Ultra Relativistic Quantum Molecular Dynamics ), valamint  PHSD ( Parton Hadron String Dynamics ).
 Ezek a szimulációs kódok széles körben használtak nem csak itt, hanem az egész nehézion fizika területén. [ detektor: 1 ]
\section{Mini projekt: $\Phi$-mezon}
\par A CBM detektor egy általános célú nehézion mérési eszköz lesz, hogy az erősen kölcsönható anyag fázisdiagramját vizsgálni lehessen. A rezonanciák nagyon fontosak, hogy a sűrű anyagot vizsgálni tudjuk az ütközés során. Az ilyen rezonanciák egyike ami fontos a CBM és a 
fázisdiagram vizsgálatának szempontjából pedig a $\Phi$-mezon, aminek nagyon kicsi a hadronokra vett hatáskeresztmetszete így eléggé valószínűtlen, hogy kölcsönhat a nagy mennyiségű hadronnal, ami a reakció során keletkezik, vagyis jó indikátora a sűrű, kezdeti eseménynek. A $\Phi$-mezon egy $strange$ és egy $anti-strange$ kvarkot tartalmaz és a kulcsa lehet az s kvark partonikus anyagban lévő keletkezésére. A $\Phi$-mezon $K^{+}, K^{-}$ párokra bomlik nagyjából 50$\%$-os eséllyel és egyebekre (pl. dileptonokra is). A közepes élettartama nagyjából $1.55\cdot10^{-22}$ s tehát még a TOF falat sem éri el, csak a bomlástermékei lesznek detektálva már jóval azelőtt is. A tömege $1.019$ MeV ami a kaonok invariáns tömegével kifejezve egy rezonancia csúcsként látható az ütközés/szimuláció után kinyert 
adatok között. Ahol a kaonok invariáns tömege:
\begin{equation*}
M_{KK} = \sqrt{(E_{1}+E_{2})^{2} - (\underline{p}_{1} + \underline{p}_{2})^{2}}
\end{equation*}
\par Én a PHSD adatait vizsgáltam, amin lefuttattam a CBM szimulációt. Egy Au+Au centrális ütközést vizsgáltam $10$ GeV-es bombázó energián.  A CBM szimuláció kimenetét a cbmROOT-tal rekonstruáltam. Több mint 5 millió esemény szerepelt a kezdeti $.root$ fájlban amit a szimulációhoz használtam.
\par A hisztogramokon az x-tengelyen a kaon párok invariáns tömege szerepel, az y-tengelyen pedig az adott energián a `beütések' száma. Egy 
apró kiugrás látható a nagy kombinatorikus háttéren nagyjából $1.02$ GeV környékén ami pontosan a $\Phi$-mezonok bomlásából adódik.
\par Igen nehéz feladat lesz hatékonyan detektálni a $\Phi$-mezonokat a CBM detektorrendszerrel. A részecskék nem csak rövid életűek, de egy hatalmas háttér is nehezíti az apró csúcs megtalálását. Ezért is kell hatalmas számú eseményt vizsgálni, hogy a csúcs
a statisztikában már látható legyen. Ennek ellenére határozottan mondhatjuk, hogy a CBM detektor képes lesz a $\Phi$-mezonok detektálására és ezáltal a strange kvark termelődésének megértésére az erősen kölcsönható anyagban.
\par A szimuláció hatékonysági mutatókat is biztosít. Mindezeket különböző részecske impulzusok esetén. A jelzett detektálás hatékonysági értékek nem túl magasak, de eléggé stabilak adott tartományokban az észleléshez.
\section{A szimulációs program}
\par Maga az ütközés a UrQMD és a PHSD programok segítségével játszódik le, a CBM szimuláció a detektor választ szimulálja, tehát az ezekből származó adatokat kapja meg kezdeti paraméternek. Ezek a modellek az ALICE, RHIC és LHC detektornak, valamint nem utolsó sorban a
CBM detektornak lettek fejlesztve.
\par Az első lépés az, hogy le kell futtatni egy Monte Carlo szimulációt, hogy képeset legyünk a `valódi' adatokat összepárosítani a keltett eseményekkel. A program ezen része arra lett tervezve, hogy kiszűrje a találatokat a detektor anyagban és olyan pontokat találjon, 
amelyek később trajektóriákká összeállíthatók.
\par A program a Geant3 (főként) és Geant4 programokat használja, hogy a részecskék anyagon való áthaladását szimulálja. Ez is a Monte Carlo szimuláció része.
\par Az első makró kimenetén tehát egy szimulációs fájl van, ami az STS és az MVD detektorok által detektált találatokat tartalmazza valamint a TOF fal és egyéb detektorok adatait is. Ezeket felhasználva lép a program a második fázisba, a rekonstrukció részhez. A rekonstrukciós 
kód először is klasztereket próbál találni az MVD detektorban, hogy megtalálja, hogy hol volt az ütközés/ütközések kiinduló pontja. Ha ezt megtalálta továbbhalad és megpróbálja lekövetni a részecske pályákat. A töltött részecskék körpályára állnak az erős mágneses tér hatására így a pontokra köríveket próbálnak illeszteni és a legjobb illesztéssel bírókat fogadják csak el (van egy százalékos határ, ami alatt hibás detektálásnak ítélik). 
\par Nyilvánvalóan, a találatok és a pályákat többször próbálja meg a program helyesen megtalálni, azért, hogy elkerülje a hibákat. Kisebb az esélye így a hibás találatnak, vagy a hibásan illesztett trajektóriának. Ennék része a digitalizáció, ami lényegében azt jelenti, hogy a szimulációs program megpróbálja a detektor választ is számításba venni. Vegyük például az STS detektort. Ennek egy szálas, hálós elrendezése van, amikor egy részecske áthalad, akkor több szálban is detektáljuk, ezek metszéspontjában van a tényleges helye. De ha egyszerre két részecske ment át `ugyan azon a ponton', akkor ezt nem láthatjuk, később a pályák illesztésénél probléma lehet. Ezért is van az, hogy ha az STS detektor több, mint 5$\%$-a detektál, akkor a rendszer lényegében nem mér, nem szerez kiértékelhető adatokat.
\par A sikeres rekonstrukció után, ami a nyers adatokból létrehozta végső soron a trajektóriákat az egyetlen visszamaradó feladat a részecske felismeres és ezek pályákhoz való párosítása. Erre egy robusztus és hatékony program áll rendelkezésre, aminek a neve KFParticleFinder.
\par Ennek a programnak a kimenete egy .root fájl, ami rengeteg részecskét és hozzájuk tartozó adatot tartalmaz, detektálási hatékonyságról, háttérről, armenteros diagramokkal, bemenő és kimenő jelekkel.
\par Ahogy korábban említettem először a Monte Carlo szimulációt kell használni valamilyen bemeneti fájllal. Ez egy .root fájl vagy egy egyszerű ASCII fájl is lehet, a szimulációs kód képes mindkettő fogadására. Egy ilyen fájlban részecske ID-k és impulzusuk található. A kimenete 
a PHSD és a UrQMD szimulációknak általában egy .root fájl, de például a HIJING sima szöveges kimenetet produkál. A CBM szimulációnál különböző függvények teszik lehetővé mindkét adattípus feldolgozását.
\par Tehát a rekonstrukció után, valamint a részecske felismerés végeztével, ha bekapcsoltuk a vizualizációt képesek vagyunk vizualizálni az eseményeket. Ehhez az $eventDisplay.C$ makrót kell futtatnunk. Ez a makró az egész CBM geometriát tartalmazza, tehát az egész 
detektort átláthatjuk vele. Megjeleníthető benne az összes trajektória és a részecskék.
\section{Nehézion fizika itthon - Részecske klaszterezés - MST (/ BFS)}
\par A Wigner Fizikai Kutatóközpontban dolgozó témavezetőmtől, Wolf Györgytől azt a feladatot kaptam, hogy az általa írt nehézion reakciós
programhoz írjak egy klaszterező programot. Ez a szimuláció a korábban említett, PHSD és UrQMD modellekhez hasonló, hazai fejlesztésű
projekt. A hadron-mag és mag-mag reakciókat transzport-egyenletek segítségével vizsgálva, a BUU-modell (Boltzmann-Uehling-Uhlenbeck modell) felhasználásával egy időfüggő, részecskék kölcsönhatását figyelembe vevő modell segítségével szimulálja ez a program.
\par Ennek kimenetén többek között szerepelhetnek bizonyos részecskék és azok momentum- és térbeli eloszlása. Detektortól függően máshogy lehet ezeket mérni. Ha olyan detektorunk van, ami csak töltött részecskéket mér, és a töltés nagyságát nem, akkor figyelembe kell vennünk, ha például térbeli (vagy impulzustérbeli) közelség miatt csak egy beütést kapunk. Így az én programom pontosan arra képes, hogy euklideszi-térben (vagy impulzustérben) klasztereket keres. Így a beütésszámra pontosabb jóslatot lehet majd adni tényleges detektor környezetben.
\par Továbbá a CBM kísérletben kutatni fogjak az egyszeres és kétszeres ritka magokat, amihez szintén szükséges a koaleszcencia szimulációja.
\par Nem tökéletesítettem még a programot, ha későbbiekben erre igény van természetesen fejlesztem. Egyenlőre hely- és impulzus-koordinátákat olvas be, majd ezután próbálja meg klaszterezni a részecskéket. A klaszterezéshez nem a legjobban ismert klaszterező algoritmust
használtam hanem az úgynevezett minimális feszítő fa ( vagy MST (Minimal Spanning Tree) a későbbiekben ) algoritmust. Ez egy gráfban a lehető legrövidebb utat találja meg. Két pont akkor van összekötve a gráfban, ha egy adott minimum távolságnál közelebb vannak. Természetesen ez a minimális távolság is a bemenetről állítható. Egy részecske egy klaszter része, ha legalább az egyik részecskéhez a klaszterben kellően közel van. 
\par Ennek az algoritmusnak talán az a legnagyobb előnye, hogy nem kell előre feltételezni, hogy hány klaszter van és azt sem, hogy azok vajon hol helyezkedhetnek el. Elméletben az algoritmus hatékonysága O($\log{m}\cdot n$) vagy O($\log{n}\cdot n + m$), ahol n a pontok
száma a gráfban, míg m az élek száma. A hatékonyság a használt adatstruktúráktól függ. Ez természetesen Prim algoritmusára igaz, vannak ennél hatékonyabb megoldások is, de számomra ez tűnt a legkényelmesebb, legmegvalósíthatóbb választásnak. Továbbá egy francia kutatócsoport Nantes-ban hasonló nehézion fizikai szimulációjában is ezt az algoritmust javasolják.
\par  Az egyik elméleti nehézség a megvalósítás során az volt, hogy az algoritmus képes legyen több klasztert formálni. Hiszen miután nem tud továbbhaladni egy klaszterben, azaz nem tud több pontot hozzáadni, ki kell venni az adathalmazból a klaszterezett pontokat. Ezután lehet csak választani egy random pontot újra, és lefuttatni az eddigi algoritmust a már redukált gráfon.
\section{Összegzés}
\end{document}